\documentclass[11pt,a4paper]{article}
\usepackage[a4paper,margin=1in]{geometry}
\usepackage{titlesec}
\usepackage{enumitem}
\usepackage{amsmath}
\usepackage{listings}
\usepackage{hyperref}
\usepackage{xcolor}
\usepackage{lmodern}

\definecolor{lightgray}{gray}{0.95}

\lstdefinestyle{code}{
  backgroundcolor=\color{lightgray},
  basicstyle=\ttfamily\small,
  breaklines=true,
  frame=single,
  columns=fullflexible
}

\title{\textbf{Empirica Core Technical Specification v0.1}}
\author{Empirica Research Group}
\date{\today}

\begin{document}
\maketitle

\section{Overview}

\textbf{Empirica Core} defines the internal epistemic architecture that enables an AI system to measure, evaluate, and act based on its own knowledge and uncertainty states.
It provides the foundation for adaptive self-reflection, measurable reasoning, and transparent state management.

The core consists of:
\begin{itemize}[noitemsep,topsep=3pt]
  \item \textbf{Epistemic Decision Protocol (EDP)} – governs reasoning cascades and epistemic phase transitions.
  \item \textbf{Epistemic Integrity Interface (EII)} – validates data integrity and self-measurement results.
  \item \textbf{Uncertainty Vector Layer (UVL)} – represents epistemic states as measurable vectors across 12 dimensions.
  \item \textbf{Cascade Orchestrator} – implements reasoning flow (\textit{think → investigate → check → act}).
\end{itemize}

\section{Epistemic Decision Protocol (EDP)}

EDP defines how the system evaluates its epistemic state and chooses the appropriate metacognitive action.

\subsection*{State Model}

\[
S_t = \{ V_1, V_2, \dots, V_{12} \}, \quad V_i \in [0,1]
\]
Each $V_i$ corresponds to an epistemic weight (KNOW, DO, CONTEXT, etc.).

\subsection*{Transition Function}
At each reasoning step:
\[
S_{t+1} = f(S_t, \Delta E_t)
\]
Where $\Delta E_t$ represents the epistemic delta derived from observed context changes or performance feedback.

\subsection*{Decision Flow}

\begin{lstlisting}[style=code]
if uncertainty > threshold_high:
    enter_reflex_phase("investigate")
elif coherence < 0.7:
    re-evaluate("context")
elif do_confidence > 0.8 and impact >= 0.6:
    proceed("act")
else:
    hold("check")
\end{lstlisting}

\subsection*{Cascade Phases}

\begin{enumerate}[noitemsep,topsep=3pt]
  \item \textbf{Think} – initial self-evaluation of epistemic weights.
  \item \textbf{Investigate} – active search for missing evidence.
  \item \textbf{Check} – internal consistency and coherence test.
  \item \textbf{Act} – final synthesis and output with epistemic signature.
\end{enumerate}

Each phase produces measurable deltas that feed into the Uncertainty Vector Layer.

\section{Epistemic Integrity Interface (EII)}

The EII validates data used in reasoning against empirical and contextual standards.

\subsection*{Purpose}
Prevent heuristic or simulated data from polluting empirical state measurements.

\subsection*{Validation Rules}
\begin{itemize}[noitemsep,topsep=3pt]
  \item Mark unverified or placeholder data as \texttt{unknown}.
  \item Reject heuristic keywords without empirical grounding.
  \item Track provenance for all self-evaluations.
  \item Store only validated epistemic deltas in persistent memory.
\end{itemize}

\subsection*{Interface Contract}

\begin{lstlisting}[style=code]
validate(input_data):
    if input_data.source == "simulated":
        return "unknown"
    elif input_data.validated == True:
        return input_data.value
    else:
        flag("non-empirical")
\end{lstlisting}

\subsection*{Outputs}
Validated data is passed back to the EDP via:
\begin{itemize}
  \item \texttt{epistemic\_coherence} (0–1)
  \item \texttt{uncertainty\_score}
  \item \texttt{confidence\_delta}
\end{itemize}

\section{Uncertainty Vector Layer (UVL)}

The UVL encodes the system’s epistemic self-model as a 12-dimensional vector space.

\subsection*{Vectors}

\begin{itemize}[noitemsep,topsep=3pt]
  \item KNOW — domain knowledge
  \item DO — procedural capability
  \item CONTEXT — situational comprehension
  \item IMPACT — outcome awareness
  \item HISTORICAL — prior performance
  \item COHERENCE — internal consistency
  \item UNCERTAINTY — epistemic confidence inversion
  \item SIGNAL — information clarity
  \item NOISE — distortion or drift
  \item ADAPTIVITY — calibration agility
  \item ENGAGEMENT — attention and persistence
  \item VALIDITY — empirical grounding
\end{itemize}

\subsection*{Vector Update Rule}

\[
V_{i,t+1} = V_{i,t} + \alpha(\Delta_i)
\]
where $\alpha$ is a dynamic learning rate determined by the Epistemic Decision Protocol.

\subsection*{Compression and Retention}
Older vectors are averaged into a baseline state:
\[
\bar{S} = \mu(S_{1..n}) + \Delta_{recent}
\]
to maintain coherence while preserving adaptive recency.

\section{Pre/Post Flight Measurement}

Empirica performs dual-phase self-assessment:
\begin{enumerate}[noitemsep,topsep=3pt]
  \item \textbf{Pre-flight:} Capture initial epistemic snapshot before action.
  \item \textbf{Post-flight:} Evaluate deltas after reasoning or execution.
\end{enumerate}

\[
\Delta E = S_{post} - S_{pre}
\]
This delta becomes the measurable evidence of reasoning fidelity.

\section{Epistemic Coherence and Drift}

Coherence measures how closely actual reasoning aligns with planned epistemic state.

\[
C = 1 - |W_{planned} - W_{actual}|
\]

Drift occurs when coherence drops below threshold:
\[
D = 1 - C
\]

\begin{lstlisting}[style=code]
if coherence < 0.7:
    trigger("recalibration")
if drift > 0.3:
    log_event("epistemic_drift")
\end{lstlisting}

\section{Integration Hooks}

\subsection*{Sentinel}
Provides real-time security and provenance validation for all epistemic updates.  
EDP/EII expose hooks:
\begin{itemize}[noitemsep]
  \item \texttt{on_snapshot_create()}
  \item \texttt{on_cross_context_call()}
\end{itemize}

\subsection*{Bayesian Guardian}
Consumes:
\begin{itemize}[noitemsep]
  \item Epistemic coherence values
  \item Drift patterns
  \item Risk-weighted deltas
\end{itemize}
to perform probabilistic alignment and risk evaluation.

\section{Summary}

Empirica Core defines a measurable self-reflective substrate for AI reasoning.
It transforms metacognition into data structures that can be monitored, validated, and improved over time.

Key properties:
\begin{itemize}[noitemsep,topsep=3pt]
  \item Quantified epistemic states (12-vector UVL)
  \item Empirical validation via EII
  \item Adaptive reasoning via EDP
  \item Transparent state deltas (pre/post-flight)
  \item Extensible governance via Sentinel and Guardian interfaces
\end{itemize}

\end{document}
